\documentclass[10pt]{article}
% \documentclass[10pt]{ctexart}
 
% \title{How To Study English In ZJU }
 
\usepackage{indentfirst}
\usepackage{capt-of}
\usepackage{hologo}

% \usepackage{appendix}
 
% for chart edition
\usepackage{chngpage}
\usepackage{graphicx}
\usepackage{array}
\usepackage{fontspec}
\usepackage{geometry}

\geometry{a4paper,scale=0.8}
\setromanfont{Times New Roman}
\setsansfont{Arial}
\setmonofont[Color={0019D4}]{Courier New}
 
 
% \title{ My Note For }
\author{Kailian Jacy}
\setcounter{page}{1}
 
\begin{document}
 
% set frontpage unlock:
\begin{titlepage}
	\newcommand{\HRule}{\rule{\linewidth}{0.5mm}}
	\includegraphics[width=8cm]{fig/zjulogo.png}\\[1cm] 
	\center 
	\quad\\[1.5cm]
	\textsl{\Large Zhejiang University }\\[0.5cm] 
	\textsl{\large School of Architecture And Civil Engineering}\\[0.5cm] 
	\makeatletter
	\HRule \\[0.4cm]
	{ \huge \bfseries How To Study English In ZJU}\\[0.4cm] 
	\HRule \\[1.5cm]
	\begin{minipage}{0.4\textwidth}
		\begin{flushleft} \large
			\emph{Author:}\\
			\@author 
		\end{flushleft}
	\end{minipage}
	~
	\begin{minipage}{0.4\textwidth}
		\begin{flushright} \large
			\emph{Supervisor:} \\
			\textup{None}
		\end{flushright}
	\end{minipage}\\[3cm]
	\makeatother
	{\large An Assignment Not for Homework but for better future}\\[0.5cm]
	{\large \emph{No Class Number here}}\\[0.5cm]
	{\large \today}\\[2cm] 
	\vfill 
\end{titlepage}

 
\setlength{\parindent}{2em}
 
% \twocolumn[{%
% \maketitle
% \begin{center}
%     \centering
%     \includegraphics[width=.9\linewidth,height=4cm]{example-image-golden}
%     \captionof{figure}{Teaser Image}
% \end{center}%
% }]
\maketitle
 
% \abstract
 
% \keywords ,
 
\section{Terminal Test}

Let's take a look at the FINAL TEST of English 3. I'll illustrate these question type one by one.
\subsection{Listening test}
\emph{ALL READ ONCE}. The listening test is a little bit like English contest I've took part in. You can only hear the question after the passage finished. You may:
\begin{itemize}
    \item lose a lot of detailes when listening, not knowing what to pay attention to. 
    \item forget the infomation afterwards. And the more you hesitate, more indecisive you are. 
\end{itemize}
So, I'm keenly to suggest that, you adopt my methods and try them all the time, when you are makeing English listening practice.
\begin{itemize}
    \item BEFORE YOU START. Read through the options on your paper and forcast: what you are to hear and what should be paid attention to. Like, some figures , and some questions and answers. Once you get yourself alert to them, you can actually make it.
    \item DON'T CHANGE YOUR ANSWER EXCEPT THAT YOU HAVE PERSUASIVE EVIDENCE. You are always changing right answers into wrong ones, trust me.
    \item And, last but not least, DROWN YOURSELF INTO EVERY PARCTICE!
\end{itemize}

\emph{News 1 : Get full marks when practicing.} failed to catch all the message the news conveyed. BUT the trick of having a guess ahead of listening is proved to be effective. 
The questions can be divided into three catergories:
\begin{itemize}
    \item \emph{CAUSE AND REASONS.} you just need to know the outline of the course. So, oay attention to what happens in the news. Get the main info.
    \item \emph{THE FOLLOWING: WHICH IS WRONG.} You have two ways to get this point: mark the option when you hear one; hear the whole story and choose the impossible one. To reduce the difficulty Level, there is probably an impossible option.
\end{itemize}
try again.

\subsection{About Text book.}
In English 3, book 1 and 2 are only used in small quiz. They are taking a atom part of your grade, trust me. You may study the two books with care, but they are only to reflect slightly in your grade. And in final test, \emph{THE ARTICLE } are not important relatively. You are only to use them in listening test D. And that can be made up with your listening skills. How many blanks you think can be get through your memrory but not listening?

In contrast, words are always important to remember. They are taking up a huge part of your test. Let's see them bellow.

\subsection{Vocabulary}
\emph{PART IV Vocabulary: Sinle choice, words and expressions are from your text book}. 
\emph{PART V Sentence Completion. Given the initials and chinese meaning, you are to fill the complete word}. \par
These two questions are requiring high on vocabulary. Single choice are always about \emph{YOUR VOCABULARY SCALE} for most , only several about vocabulary discrimination. Your first goal should be to recognize all the words in the vocabulary lists, and second to memorize words for LEVEL Four And Six, only third place to tell them apart.
Beside this, single choice also examine your grammar ability (2 in 20). But Grammar, I think, should be a basic requiment for us to get good grade, that should not be regard as a difficulty to solve in this terminal period.\par
And as you can see, you are required to give the word by chinese meaning in question 5 . So I highly suggest a \emph{Silent writing from Chinese meaning}. That could help you to use them into your writing, at the same time .
\textbf{Completion: Got 2 errors when practicing. } One not reviewed yet (tonight on schedule) and one I didn't get its full meanings, which suggests that I'm not familiar enough with these words. 
\textbf{Single choice: Got 1 errors when practicing. } For Grammar reasons.

\subsection{Banked Cloze. }
\emph{Choose ten words out of fifteen, and choose the right form of you words.} 
Ordinarily, you can fill 3-5 words in easily, and troubled with the last 2 or 3 words powerlessly. To save your time, you should read the passage three times. Each time you just put the absolute 3 or 4 words. Once you get the most of them right, you can get these words in relatively easily.
And Also, that suggest up to memorize the various form of words. 

\textbf{get full marks at practice. No unfamiliar words.} Should 'conflict' be plural form ? nothing prevent it from a countable noun to be plural.

\subsection{writing}
Writing should be about 150 words. My teacher emphasized \emph{'do not exceed 200 words'}. These are suggestions :
\begin{itemize}
    \item DON'T TAKE THE WRITING TO SERIOUSLY. Usually, the article you spend a lot of time on gets 13 points (I believe some not ) while the rubbish you complete in 10 minutes gets 11 to 12 points. Compared to this, you'd better to check on your former answers. 
    \item YOU ARE SUGGESTED TO USE THE WORDS IN TEXTBOOK. Most students are not serious about English 3 class, but you learned much here and use them in your writing, that make sense of your commitment to your teacher.
    \item The subjects are always from the book and changed a bit. 
\end{itemize}
So, as you can see , there are a lot of clues compelling you to learn the vocabulary well. It's highly recommended to include the words in many ways: you can list the words around a subject of a unit, you can recipe similar words...

\section{Plan for tomorrow revision}
\textbf{For Words}: include them in CHINESE MEANINGS. Get up early and read them aloud, get your days start; List out unfamiliar words for third time reivision right before get into the examination room. (around 20-30)
This time, BE CAREFULE ABOUT SPELLING THE WORDS.\par
\textbf{For Writing}: try for one subject. Just one to get prepared.




 
% add charts and appendix.
% \begin{appendices}
 
% % \chapter{Your Appendix Name}
 
% \begin{table}[!bhp]
% \centering
% \begin{adjustwidth}{-7cm}{0cm}
 
%  \begin{tabular}{|p{.2\textwidth}| p{.2\textwidth}| m{.5\textwidth}|}
 
% \hline
% item & item &
 
% % \bottomrule
% \end{tabular}
% \label{tab:example}
%     \caption{example chart.}
% \end{adjustwidth}
% \end{table}
 
%   \end{appendices}
 
\acknowledgements
Thank you for reading
 
\bibliography{}
 
\end{document}