\documentclass[10pt]{article}
\usepackage[CHN]{iscv}
\usepackage{capt-of}
\usepackage{hologo}
\bibliographystyle{ieeetr}

\title{ISCV 论文模板}
\author{第一作者 \and 第二作者}
\setcounter{page}{1}

\begin{document}

% \twocolumn[{%
% \maketitle
% \begin{center}
%     \centering
%     \includegraphics[width=.9\linewidth,height=4cm]{example-image-golden}
%     \captionof{figure}{Teaser Image}
% \end{center}%
% }]
\maketitle

\abstract
摘要写在这里。摘要应简明地概括文章的主要内容。

\keywords 最多,5个,关键词

\section{导论}
科研工作的成果常以论文的形式发表在学术会议或学术刊物上,论文是科研工作者相互交流的重要媒介。
因此,文献写作是科研工作的重要环节。各类学术文献都有自己的标准排版格式,本文介绍的 ISCV 模板
提供了 ISCV 月刊的标准排版。通过 \LaTeX{} 和 ISCV 宏包配合,结合基础的 \LaTeX{} 排版知识,即可
写出符合 ISCV 标准的文献。

\section{基本用法}
ISCV 宏包的使用方法非常简单,只需将文档类型设为 \texttt{article}, 字号设置为 10pt,再引用 \texttt{iscv} 宏包即可,如下所示:
\begin{verbatim}
\documentclass[10pt]{article}
\usepackage[CHN]{iscv}
\end{verbatim}
注意,\texttt{CHN} 选项表明了该文档使用中文排版,若需要排版英文文献,则使用
\begin{verbatim}
\usepackage[ENG]{iscv}
\end{verbatim}
为了支持中文排版,你需要使用 \hologo{XeLaTeX} 来编译 tex 工程。如果你使用 Overleaf, 只需在工程菜单中将编译器(Compiler)设置为 XeLaTeX 即可。

\section{排版规范}
ISCV 审稿人委员会会对每一篇 ISCV 投稿进行审稿。投稿文章应以中文或英文写作,并且排版需要 ISCV 规范。本文即为 ISCV 排版规范的示例文档。
委员会提供了 \LaTeX{} 模板,建议使用 \LaTeX{} 排版 ISCV 投稿文献。如果使用 MS Word,投稿人需要自行排版以遵循 ISCV 相关规范。

ISCV 投稿文献可分为以下三类:
\begin{itemize}
    \item 论文: 最多 4 页
    \item 技术报告: 最多 2 页
    \item 新闻稿: 1 页
\end{itemize}
页数统计包含文章所有图表,但不包含参考文献。如果有补充材料,包括图片、详细分析、视频等,可以发表在论坛上。
若投稿被接收,作者将会受邀在研讨会上讲解自己的工作。

所有投稿需要使用 B5  (17.6 cm $\times$ 24 cm) 纸,文章内容应在纸张中心的 {14 cm $\times$ 20 cm} 范围内,即页面上下边距为 2cm,左右边距为 1.8cm.
投稿的论文和技术报告需要双栏排版。

\subsection{图表}
文章图表需要居中放在每一栏中,并按其出现的顺序编号。每一个图、表下方应包含标题和编号 (示例见图~\ref{fig:example}或表~\ref{tab:example})。
注意图表中的文字不要过大或过小。
对于需要横跨双栏的图表,可以使用 \texttt{figure*} 或 \texttt{table*}环境, 跨栏图表应使用\texttt{[t]} 或 \texttt{[b]} 选项将其置于页面顶端或底端。

\begin{figure}[h]
\centering
\includegraphics[width=.5\linewidth]{figs/fig1.pdf}
\caption{示例图片} 
\label{fig:example}
\end{figure}

\begin{table}[h]
\centering\begin{tabular}{ccc}
\toprule
Net & top1 & top5 \\
\midrule
ResNet & 7\% & 5\% \\
VGG & 10\% & 7\% \\
ShuffleNet & 15\% & 10\% \\
\bottomrule
\end{tabular}
\caption{示例表格}
\label{tab:example}
\end{table}

\subsection{公式}
文中独立出现的文字应居中排版,并按其出现的顺序编号。编号应向右对齐,并由圆括号包围。使用 \LaTeX 排版时会自动生成公式编号。
在引用公式时,使用 \verb|\eqref{eq:label}| 产生正确的引用格式,如式~\eqref{eq:example}。
\begin{equation}
\label{eq:example}
	    PA + A'P - PBR^{-1}B'P + Q  =  0
\end{equation}

\section{生成 PDF 文件}
最终文献以 PDF 为标准格式。使用 \LaTeX{} 排版,并用 \hologo{XeLaTeX} 编译则会自动生成符合要求的 PDF 文件。建议使用 Overleaf 处理 \LaTeX{} 文件。

\section{参考文献}
参考文献列表应与引用顺序一致。文中引用文献应使用方括号形式,例如文献~\cite{ref01}和文献~\cite{ref01,ref02}。

\acknowledgements
致谢应置于文章末尾。

感谢您阅读本文!

\bibliography{iscv}


\end{document}
